\documentclass[10p]{report}
\usepackage[utf8]{inputenc}
\usepackage[greek]{babel}
\usepackage{alphabeta}
\usepackage[LGR, T1]{fontenc}
\usepackage{textpos}
\usepackage{dejavu}
\usepackage[left=1.2in, right=1.2in]{geometry}
\usepackage{amsmath}
\usepackage{listings}
\usepackage[hidelinks]{hyperref}
\usepackage{changepage}

\begin{document}

% Title
%\begin{textblock}{7.5}(1.99,-1.5)	
\begin{center}
\textbf{\Large{Ενσωματωμένα Συστήματα \\ Πραγματικού Χρόνου}}\\
\textbf{\large{Αναφορά \latintext{Εργασία 3}}} \\
\normalsize{Παρασκευόπουλος Ιάσων (ΑΕΜ 8410)}
\end{center}
%\end{textblock}

\section*{\latintext{Server}}

Η υλοποίηση του \latintext{server} ακολουθεί την λογική όπου με κάθε εισερχόμενο request 
δημιουργείται ένα νέο thread για την διαχείρισή του μέχρι ένα συγκεκριμένο πλήθος
από requests. Στην περίπτωση που υπάρχουν ταυτόχρονα ενεργά threads όσο το όριο 
τότε τα επόμενα requests μπαίνουν σε μία ούρα και μόλις τελειώνουν κάποια από τα 
ενεργά παίρνουν την θέση τους με την σειρά που μπήκαν. Είναι παρόμοια 
η λογική με thread pool server *LINK* με την διαφορά ότι αντί να υπάρχουν από την αρχή 
τα προκαθορισμένα threads και να περιμένουν, δημιουργούνται με τον ερχομό των requests και 
αν φτάσουν στο όριο όταν τελειώνει κάποιο, ένα άλλο παίρνει την θέση του χωρίς να 
ξεπερνάει το όριο το πλήθος τους. 

Πιο συγκεκριμένα 
 
\section*{\latintext{Client}}

\section*{Έλεγχος ορθότητας}

\section*{Μετρήσεις και σχόλια}

\end{document}

